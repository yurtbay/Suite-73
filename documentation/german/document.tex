% % % % % % % % % % % % % % % % % % % % % % % % % % % %
%                                                     %
%   Suite73 Documentation                             %
%                                                     %
%   Author:    SCHWARZER, Andre                       %
%              von RHEIN, Christopher                 %
%              YURTBAY,   Emre                        %
%                                                     %
%   Language:  German                                 %
%                                                     %
%   Date:      22. March 2014                         %
%                                                     %
% % % % % % % % % % % % % % % % % % % % % % % % % % % %

\documentclass[a4paper,12pt]{article}

\usepackage{amssymb} % Mathe
\usepackage{amsmath} % Mathe
\usepackage[utf8x]{inputenc} % Umlaute
\usepackage[ngerman]{babel}  % Umlaute
\usepackage[T1]{fontenc}     % Umlaute im PDF
\usepackage[margin=2.5cm]{geometry} % Layout
\usepackage{booktabs}
\usepackage{times}
\usepackage{colortbl}
\usepackage{caption}
\usepackage{float}
\usepackage{multirow}
\usepackage{cite}
\usepackage{color}



% links
\usepackage[colorlinks=true, linkcolor=blue, citecolor=blue,]{hyperref}  

% glossar
\usepackage[nonumberlist]{glossaries} 

\usepackage{hyperref}

\ifx\pdftexversion\undefined
\usepackage[dvips]{graphicx}
\else
\usepackage[pdftex]{graphicx}
\DeclareGraphicsRule{*}{mps}{*}{}
\fi


\newglossaryentry{CMS}
{
  name=CMS,
  description={Content Management System}
}

\newglossaryentry{ERP}
{
  name=ERP,
  description={Enterprise Resource Planning}
}
\addto\captionsngerman{
 \renewcommand\glossaryname{}}
\renewcommand*{\glstextformat}[1]{\textit{\textcolor{black}{#1}} }
\makeglossary

% % % % % % % % % % % % % % % % % % % % % % % % % % % %
% Variablen                                           %
% % % % % % % % % % % % % % % % % % % % % % % % % % % %
\newcommand{\authorName}{André SCHWARZER, Christopher von RHEIN, Emre YURTBAY}
\newcommand{\auftraggeber}{Hochschule Bochum - University of Applied Sciences}
\newcommand{\auftragnehmer}{Studentisches Projekt}
\newcommand{\projektName}{Suite73 - All-In-One Data Management}
\newcommand{\documentation}{Dokumentation}
\newcommand{\tags}{\authorName}
\newcommand{\glossarName}{Erläuterung zu Begriffen und Abkürzungen}
\title{\projektName}
\author{\authorName}
\date{\today}

\definecolor{hellgrau}{rgb}{0.95,0.95,0.95}
\definecolor{koenigsblau}{rgb}{0.00, 0.09, 0.45}

% % % % % % % % % % % % % % % % % % % % % % % % % % % %
% PDF Meta Informationen                              %
% % % % % % % % % % % % % % % % % % % % % % % % % % % %
\hypersetup{
  pdfauthor   = {\authorName},
  pdfkeywords = {\tags},
  pdftitle    = {\title~(Doku)}
}

% % % % % % % % % % % % % % % % % % % % % % % % % % % %
% Dokument                                            %
% % % % % % % % % % % % % % % % % % % % % % % % % % % %
\begin{document}
    \pagenumbering{roman}
    \begin{titlepage}
\newgeometry{left=7.5cm} 
\pagecolor{koenigsblau}
\noindent
\includegraphics[width=2cm]{73.png}\\[-1em]
\color{white}
\makebox[0pt][l]{\rule{1.3\textwidth}{1pt}}
\par
\noindent
\textbf{\textsf{Hochschule Bochum}} \textcolor{hellgrau}{\textsf{University of
Applied Sciences}}
\vfill
\noindent
{\huge \textsf{\documentation}}
\vskip\baselineskip
\noindent
\textsf{\today}
\end{titlepage}
\restoregeometry % restores the geometry
\nopagecolor% Use this to restore the color pages to white
    \setcounter{page}{2}
    \setcounter{tocdepth}{4}
    \setcounter{secnumdepth}{4}
    \tableofcontents
    \clearpage
    \pagenumbering{arabic}
 
	\begin{abstract}
		Abstrakte Beschreibung \ldots
	\end{abstract}	
	\clearpage
    
    \printglossaries
	\clearpage
 
	\section{Lastenheft}
 	
		\subsection{Einleitung} %A
 		Als Basis des Projekts wird eine Datenverwaltungssoftware entwickelt, die
eine Benutzerrechteverwaltung sowie eine allgemeine Datenverwaltung bietet.
Die Benutzerrechteverwaltung hat zwei Zuständigkeiten:
\begin{enumerate}
	\item Zugriffsrechteverwaltung auf die Verwaltungssoftware
	\item Zugriffsrechteverwaltung auf die Produkte (CMS, ERP, \ldots)
\end{enumerate}
 		
		\subsection{Ist-Zustand} %C
 		Der aktuelle Markt stellt keine einheitliche Lösung für die Verarbeitung und Verwaltung von Inhalten, Mitarbeiter, Kunden, Händler und andere, für einen Unternehmen relevante, Positionen zur Verfügung, die für ein kleines oder mittelständisches Unternehmen bezahlbar sind.\\
\\
Darüber hinaus muss man momentan aus einer Sammlung von diversen Programmen ein System selbst zusammenstellen, welches den gegebenen Anforderungen entspricht.\\
Hierbei kann es zu erhöhtem Kosten- und Zeitaufwand durch die Administration und Migration der Daten kommen.\\
Des Weiteren fehlt in den Unternehmen oft das nötige Fachwissen, um sich ein derartiges System zusammenstellen zu können.

 		
		\subsection{Soll-Konzept} %E
 		Beschreibung des Soll-Konzepts
	Der jetzige Markt erlaubt es Klein- sowie Mittelstandsunternehmen nicht, eine Data Management Lösung wahrnehmen zu können, da solche Programme für Großunternehmen entwickelt und nur als SAAS mit beschränkter Bandbreite angeboten werden.
	Die Suite 73 soll daher diesen Unternehmen die Möglichkeit bieten, eine All-In-One Data Management Lösung kostengünstig nutzen zu können.
	Die Base 73 soll ein CMS enthalten und alle Funktionen frei anbieten. Erweiterung von CRM, ERP bis hin zum Onlineshop sollen optional kostengünstig zu erwerben sein.
	Zu den Inhalten sollen SEO & SEM's gegeben sein, ein Frontend und ein Backend  mit den entsprechenden Funktionen (Drag & Drop)sollten auch bereit stehen.
 		
		\subsection{Schnittstellen}
 		\input{lastenheft/schnittstellen}
 		
		\subsection{Funktionale Anforderungen}
 		\subsubsection{Base73}
    \paragraph{Zugriffsverwaltung}
        \begin{itemize}
            \item Benutzerverwaltung
            \item Konzept verschiedener Rollen (Nutzer, Moderator, Administrator, \ldots)
            \item OAuth Berechtignungen verwalten
        \end{itemize}
    \paragraph{Datenverwaltung}
		\begin{itemize}
			\item abstrakter Datahandler
                \begin{itemize}
                    \item definiertes Datenmodell als JSON Datei
                    \item Import sowie Export von Datenmodellen
                    \item Import sowie Export von Daten
                \end{itemize}
            \item REST API
                \begin{itemize}
                    \item kommuniziert mit MySQL Datenbanken
                \end{itemize}
		\end{itemize}
\end{enumerate} 


\subsubsection{CMS73}
	\paragraph{SEO \& SEM}
		\begin{itemize}
			\item SEF URLs
			\item OpenSearch
			\item RSS Feed (geringe Priorität)
			\item Rich Snippets (vgl.: https://schema.org/)
		\end{itemize}
	
	\paragraph{Inhalte (Frontend)}
		\begin{itemize}
			\item Artikel Layout
			\item Blog Layout
			\item Galerien!
			\item Formulare (z.B.: Kontakt, Umfragen, Bewertungen, …)
			\item Kommentare
			\item Social Media
			\item Kalender (mit Interaktionsmöglichkeiten wie z.B. Terminbuchung, …)
		\end{itemize}
		
	\paragraph{Inhalte (Backend)}
		\begin{itemize}
			\item Media + Downloadmanagement
			\item Templatemanagement
			\item Möglichkeit der Installation von Erweiterungen (Shop, …)
			\item Beiträge/Inhalte/Artikel
			\begin{itemize}
				\item Beitragsverlauf (Versionshistory)
				\item Drag \& Drop
				\item Menüzuweisung
				\item Syntaxhighlighting
				\item Widgets (Mediaplayer, 3D Rendering, …)
			\end{itemize}
			\item Automatische Menüs
			\item Formulargenerator
			\item Sprachmanagement
			\item Zugriffsstatistik
		\end{itemize}
 		
		\subsection{Nichtfunktionale Anforderungen}
            \subsubsection{Benutzbarkeit}
            Die Suite73 soll schon zu Beginn mehrsprachig (deutsch und englisch) angeboten werden, Spracherweiterungen können später über die Benutzeroberfläche hinzugefügt werden.\\
Des Weiteren soll Suite73 per Drag \& Drop dem Benutzer eine einfache und intuitive Bedienung ermöglichen
            
            \subsubsection{Zuverlässigkeit}
            Um die Zuverlässigkeit zu gewährleisten wird das Produkt an ein Rechenzentrum angebunden (höhere Bandbreite).\\Des Weiteren wird die vorhandene Serverarchitektur und Rechenzeit optimierte Software für sichere und stabile bzw. authentifizierte zugriffe zulassen.
            
            \subsubsection{Effizienz}
            Die Suite73 soll dem Nutzer schnelle Zugriffszeiten erlauben und über alle Funktionen hinweg konsistent bedienbar sein.
            
            \subsubsection{Änderbarkeit}
            Die Suite73 wird modular aufgebaut. Dadurch besteht zu jedem Zeitpunkt die Möglichkeit einzelne Teile des Projektes zu ändern ohne Änderungen am gesamten Projekt vornehmen zu müssen.\\
\\
Gewisse Teile, wie z.B. Schnittstellen zu bestehenden Datenbanken werden ebenfalls ausgelagert, da sich diese in der Regel nach Beginn eines Projektes nicht weiter Ändern.
            
            \subsubsection{Übertragbarkeit}
            Serverseitig wird das Projekt nur auf unseren Servern laufen, die Übertragbarkeit ist demnach hier nebensächlich.\\
\\
Clientseitig wird lediglich ein aktueller Browser benötigt.
            
            \subsubsection{Wartbarkeit}
            Durch den modularen Aufbau des Projektes ist eine hohe Wartbarkeit gewährleistet.\\
Der Kunde hat die Möglichkeit individuelle Funktionen, Wünsche und Designs in das Projekt einzubringen.\\
        
        \subsection{Risikoakzeptanz}
 		Die Risikoakzeptanz für die identifizierten möglichen Schadensfälle wird beispielsweise in Form einer Risikoakzeptanzmatrix dokumentiert. Die Matrix ist eine Vorgabe des Auftraggebers, in der er festlegt, bei welcher Schadensklasse und welcher Eintrittswahrscheinlichkeit er welche Risikoklasse akzeptiert.
--folgt von Prof. Koehn--
        
        \subsection{Entwicklungszyklus}
 		Jedes Modul der Suite73 wird nach Bedarf angepasst, sodass zu jedem Zeitpunkt eine individuell angepasste Version f�r den Nutzer zur Verf�gung steht.
        
        \subsection{Systemarchitektur} %A
 		Server:
\begin{itemize}
    \item SSH
	\item Apache 2 (\textit{mod\_ssl aktiviert})
	\item PHP 5
	\item MySQL 5
\end{itemize}
Client: 
\begin{itemize}
	\item aktueller Browser
    \item \textit{optional:} WebKit
\end{itemize}
        
        \subsection{Lieferumfang} %C
 		Die Suite73 beinhaltet:

\begin{enumerate}
	\item Base73 
	\begin{itemize}
		\item Benutzerverwaltung
		\begin{itemize}
			\item Konzept verschiedener Rollen
		\end{itemize}
		\item Datenverwaltung
		\begin{itemize}
			\item abstrakter Datahandler
			\item kommuniziert mit MySQL Datenbanken
		\end{itemize}
	\end{itemize}
	\item Standardtemplates für installierbare Systeme
	\item CMS auf Grundlage von Base73
\end{enumerate} 






%vonRhein
        
        \subsection{Abnahmekriterien} %A
 		"Folgen von Prof. Dr. Köhn"
 	\clearpage

	\section{Pflichten}
 	Hier die Pflichen

\end{document}

